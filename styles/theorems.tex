%%%%%%%%%%%%%%%%%%%%%%%%%%%%%%%%%%%%%%%%%%%%%%%
%%%%%%%%%%%%%%%%%%%%%%%%%%%%%%%%%%%%%%%%%%%%%%%
%
%This is where one would tell \LaTeX{} how to format Theorems, Definitions, etc. and also %indicate the environment names. You need the amsthm package (loaded in the woosterthesis %class) in order for these commands to work.
%
%%%%%%%%%%%%%%%%%%%%%%%%%%%%%%%%%%%%%%%%%%%%%%%
%%%%%%%%%%%%%%%%%%%%%%%%%%%%%%%%%%%%%%%%%%%%%%%

% an example of defining your own theoremstyle
%\newtheoremstyle{break}% name
%  {\topsep}%      Space above
%  {\topsep}%      Space below
%  {\itshape}%         Body font
%  {}%         Indent amount (empty = no indent, \parindent = para indent)
%  {\bfseries}% Thm head font
%  {.}%        Punctuation after thm head
%  {\newline}%     Space after thm head: " " = normal interword space;
%        %       \newline = linebreak
%  {}%         Thm head spec (can be left empty, meaning `normal')
\newtheoremstyle{scthm}{\topsep}{\topsep}{\itshape}{}{\bfseries}{.}{ }{}
\theoremstyle{break}
\newtheorem{thm}{Theorem}[chapter]%number theorems within chapters 
\newtheorem{cor}[thm]{Corollary}%by using [thm] we are numbering these environments with the theorems. 
\newtheorem{lem}[thm]{Lemma}
\newtheorem{prop}[thm]{Proposition}
\theoremstyle{definition}
\newtheorem{defn}[thm]{Definition}
\theoremstyle{remark}
\newtheorem{rem}[thm]{Remark}
\renewcommand{\therem}{}
\newtheorem{ex}[thm]{Example}
\theoremstyle{plain}
\newtheorem{note}[thm]{Notation}
\renewcommand{\thenote}{}
\newtheorem{nts}[thm]{Note to self}%use to remind yourself of things yet to do
\renewcommand{\thents}{}
\newtheorem{terminology}[thm]{Terminology}
\renewcommand{\theterminology}{}