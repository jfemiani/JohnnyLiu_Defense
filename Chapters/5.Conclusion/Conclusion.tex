\chapter{Conclusion}
In this thesis, a method to estimate the thickness and to refine the trajectory of a ribbon-like feature in \ac{VHR} aeral orthoimagery has been presented. 
We were able to obtain both high qualitative and quantitative results for segmenting walking paths in aerial imagery by refining \ac{OSM} features. 
The \ac{DPST} has only a few parameters with clear interpretations.
The proposed approach does not rely on extensive training on annotated data, but instead learns to fit the underlying imagery based on a trimap that is specified by parameters $\MinRadius{}, \MaxRadius{}, \MaxDistance{}$ which values that are easily determined based on the task, and are unlikely to require adjustment. 

We evaluated this work by comparing it mainly to approaches for interactive image segmentation. 
Since most automatic approaches for extracting street-maps focus on extracting coarse trajectories (which we take as input) rather than on precise alignment and width estimation. 
We focused on modeling footways which are small, thin, easily occluded, and often modified due to construction.
We believe that this approach is a natural tool to improve segmentation by post-processing automatically extracted street networks, and also by enabling fast interactive annotation of aerial images by refining hand-entered paths such as those found in \ac{OSM}.

\section{Summary}
Chapter 2 introduces a few important background knowledge for better understanding of this thesis defense. We go through \ac{VHR}, \ac{OSM}\cite{OpenStreetMap}, \ac{CRF}\cite{MAL-013}, \ac{GMM}\cite{sridharan2014gaussian} and \ac{DP}\cite{bellman2013dynamic} to understand the basic of our approach.   

Chapter 3 shows the experiments that we researched on similar topics on feature extraction from image and sidewalk segmentation method. Mainly, we compared with segmentation tools, such as grabcut\cite{Rother2004-ou}, slic\cite{Achanta:149300}, and active contours \cite{Kass88snakes:active}. These three methods had fairly performance on segmentation sidewalk feature from input. Different but related methods were also applied such as road extraction\cite{road_detect}.

Chapter 4 carefully considers a hypothesis for our dynamic programming approach. Introduce the idea to segment ribbon-like feature by converting it's original shape into ribbon-shape. Also we developed a mathematics solution and a lattice representation of the problem, to help us build our dynamic programming solution better in the chapter 6.

Chapter 5 shows the publication that submitted base on our approach. We reproduce the hypothesis with more detail demonstration for the approach and involved algorithms. 

Chapter 6 brings a four-step approach in order to generate precise boundaries for given sidewalk, locate sidewalk geometric information, generating ribbon-image, applying density estimation function and applying our dynamic program algorithm with width control. Output from the approach would be compared from the other methods.

