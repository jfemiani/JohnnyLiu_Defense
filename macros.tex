% Various macros used in the paper



% The input image (RGB) 
%\newcommand{\image}{M\xspace}
\def\InputImage{\ensuremath\vec{X}}

% An indicator random variable
\newcommand{\indicator}[1]{1_{{#1}}}

\newcommand{\Color}{\lambda}

%
\newcommand{\pixelnll}{\mathtt{E_{pix}}\xspace}

%
\newcommand{\rownll}{\mathtt{E_{row}}\xspace}

%
\newcommand{\ribbonnll}{\mathtt{E_{pth}}\xspace}

\def\vec#1{\ensuremath{\mathbf{#1}}}

% The vector x (bold)
\newcommand{\x}{\mathbf{x}\xspace}

% A reference to a figure -- the convention will vary based on where we publish
% \newcommand{\figref}[1]{\figurename~\ref{#1}}


\def\figref#1{Figure~\ref{#1}}

\def\RibbonImage{\ensuremath\vec{W}}

\def\InputTrajectory{\vec{P}}
\def\OutputTrajectory{\ensuremath{\tau}}
\def\OutputRadius{\ensuremath{\rho}}

\def\MaxRadius{\ensuremath{k_f}}   % Maximum radius of the ribbon (est)
\def\MinRadius{\ensuremath{k_n}}   % Minimum radius of the ribbin (est)
\def\MaxDistance{\ensuremath{k_W}} % Maximim distance from the center of the ribbin (n = 2*\MaxDistance + 1)

\def\GrabCut{GrabCut}
\def\ActiveContours{Active Contours}
